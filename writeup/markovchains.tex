\documentclass{amsart}
\usepackage{amsmath}
\usepackage[utf8]{inputenc}
\usepackage{amsthm}
\usepackage{csquotes}
\usepackage{todonotes}
\usepackage{verbatim}
\usepackage{amssymb}

\usepackage{tikz}
\usepackage{tikz,fullpage}
\usetikzlibrary{arrows,%
                petri,%
                topaths}%
\usepackage{tkz-berge}
\usepackage[position=top]{subfig}




\newtheorem{thm}{Theorem}[section]
\newtheorem{prop}[thm]{Proposition}
\newtheorem{lem}[thm]{Lemma}
\newtheorem{cor}[thm]{Corollary}


\theoremstyle{definition}
\newtheorem{definition}[thm]{Definition}
\newtheorem{example}[thm]{Example}
\newtheorem{theorem}{Theorem}

\newtheorem{lemma}[theorem]{Lemma}

\theoremstyle{remark}

\newtheorem{remark}[thm]{Remark}

\newtheorem{corollary}{Corollary}[theorem]

%%%
%%% The following, if uncommented, numbers equations within sections.
%%% 

\numberwithin{equation}{section}


\title{Markov Chains}
\author{Joseph Tobin}
\date{17 November 2017}

\begin{document}

\maketitle

%include abstract

%5.17 - end
%presentation the week after break
%email rough draft over beak
%lookup and use jordan canonical form without proof


\section{Introduction}

\begin{lemma}
hello world

\end{lemma}

\begin{theorem}[Theorem 5.18 pg. 298]
Let $A \in M_{n \lambda n} (C)$ be a matrix in which each entry is positive and let $\lambda$ be an eigenvalue of $A$ such that $| \lambda | = \rho(A)$.
Then $\lambda = \rho(A)$ and $\{ u \}$ is a basis for $E_{\lambda}$, where $u \in C^n$ is the column vector in which each coordinate equals $1$.

\end{theorem}

\begin{proof}
First, note that because $A$ has all postitive values and $u$ has all positive values, then $Au$ has all postive values. Therefore, because $\lambda u = Au$ has all positive values we can conclude $\lambda > 0$ and $\lambda = |\lambda| = \rho(A)$

\todo{how do we know 1 is an eigenvalue of A? we don't know that it's a transition matrix?}

Now let $v$ be an eigenvector of $A$ corresponding to $\lambda$ with coordinates $v_1, v_2, \ldots, v_n$.  
Then let $v_k = max(|v_1|, |v_2|, \ldots |v_n|)$ and $b = |v_k|$.

Then $$ |\lambda| b = |\lambda| |v_k| = |\lambda v_k| $$
But if $\lambda$ is an eigenvalue of $A$, then $Av = \lambda v$ and thus $\forall 1 \leq i \leq n \lambda v_i = \sum_{j = 1}^n A_{ij}v_j$.
Thus

$$ |\lambda v_k| = | \sum_{j = 1}^n A_{kj}v_j | $$

By the triangle inequality and then multiplication rules,

$$ | \sum_{j = 1}^n A_{kj}v_j | \leq \sum_{j=1}^n |A_{kj}v_j| = \sum_{j=1}^n |A_{kj}| |v_j| $$

Since we know $b = |v_k| \geq v_i \forall 1 \leq i \leq n$ and similarly $\rho(A) \geq  \rho_i(A) \forall 1 \leq i \leq n $, we know 

$$ \sum_{j=1}^n |A_{kj}| |v_j|  \leq \sum_{j=1}^n |A_{kj}| b = b \sum_{j=1}^n |A_{kj}| =  b\rho_k(A) = \rho_k(A)b \leq \rho(A)b $$


But since we know $|\lambda| = \rho(A)$, we know the three inequalities above are actually equalities.

\begin{enumerate}

	\item $| \sum_{j = 1}^n A_{kj}v_j | = \sum_{j=1}^n |A_{kj}v_j|$

	\item $\sum_{j=1}^n |A_{kj}| |v_j|  = \sum_{j=1}^n |A_{kj}| b$

	\item $\rho_k(A)b = \rho(A)b$

\end{enumerate}


But now we can use this to show that $\{ u \}$ is a basis for $E_{\lambda}$



\end{proof}

\begin{corollary}[Corollary 1 pg. 299]

	Let $A \in M_{n \times n}(C)$ be a matrix in which each entry is positiive and let $\lambda$ be an eigenvalue of $A$ such that $|\lambda| = \nu(A)$.
	Then  $\lambda = \nu(A)$ and $E_{\lambda}$ has dimension 1.

\end{corollary}

\begin{proof}

	Consider $A^T$.  
	We know by execercise 14 of 5.1 in \cite{friedberg2003linear} that $A$ and $A^T$ have the same eigenvalues.
	Let $E_{\lambda}, E_{\lambda}\textprime$ be the eigenspaces of $\lambda$ corresponding to $A, A^T$ respectively.
	Additionally, if $|\lambda| = \nu(A)$, then the row \todo{define nu, rho, clarify} corresponding to $\nu(A)$ is now a column in $A^T$ such that $|\lambda| = \rho(A^T)$.
	Thus $A^T$ is a matrix in which each entry is positive with an eigenvalue $\lambda = \rho(A^T)$.
	Thus by Theorem 5.18 \cite{friedberg2003linear}, the basis of $E_{\lambda}\textprime = \{ u \}$, where $u \in C^n$ is the column vector in which each coordinate contains $1$ and $\lambda = \rho(A^T)$.
	Thus $dim(E_{\lambda}\textprime) = 1$.
	But by exercise $13$ of $5.2$ of \cite{friedberg2003linear}, we know $dim(E_{\lambda}) = dim(E_{\lambda}\textprime) = 1$ and $\nu(A) = \rho(A^T) = \lambda$.
	Thus, we have shown $\lambda = \nu(A)$ and $dim(E_{\lambda}) = 1$.

\end{proof}



\begin{corollary}[Corollary 2 pg. 299]

	Let $A \in M_{n \times n}(C)$ be a transition matrix in which each entry is positiive and let $\lambda$ be an eigenvalue of $A$ such that $\lambda \neq 1$.
	Then  $|\lambda| < 1$ and the eigenspace corresonding to the eigenvalue $1$ has dimension $1$.

\end{corollary}

\begin{proof}
	We know by corollary 3 of Theorem 5.16 \cite{friedberg2003linear} that if $\lambda$ is an eigenvalue of a transition matrix, then $|\lambda| \leq 1$.
	Thus if $|\lambda| \neq 1$, then $|\lambda| < 1$.

	If $A$ is a transition matrix, then by Theorem 5.17 \cite{friedberg2003linear} we know that $1$ is an eigenvalue. 

	We also know that if $A$ is a transition matrix, then given $u$ as a column vector in which each coordinate equals 1, then $A^Tu = u$.
	But since $u$ is the column vector equal to $1$, each $u_i$ in $u$ is equal to the sum along the columns of $A^T$.
	But because $u_i = 1 \forall i$, then $\nu_i(A) = 1 \forall i$ and thus $\nu(A) = 1$.

	Thus we have an all-positive-entry matrix with $1 = \lambda = \nu(A) $ and thus by Corollary 1 of Theorem 5.18 \todo{fancy citing}, we know $dim(E_{\lambda}) = 1$.


\end{proof}

\section{Theorem 5.19}

\section{Theorem 5.20}





\section{Explanation of Film}

\bibliographystyle{alpha}
\bibliography{MathCitations} 


\end{document}